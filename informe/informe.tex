% ******************************************************** %
%              TEMPLATE DE INFORME ORGA2 v0.1              %
% ******************************************************** %
% ******************************************************** %
%                                                          %
% ALGUNOS PAQUETES REQUERIDOS (EN UBUNTU):                 %
% ========================================
%                                                          %
% texlive-latex-base                                       %
% texlive-latex-recommended                                %
% texlive-fonts-recommended                                %
% texlive-latex-extra?                                     %
% texlive-lang-spanish (en ubuntu 13.10)                   %
% ******************************************************** %


\documentclass[hidelinks,a4paper,10pt, nofootinbib]{article}
\usepackage[spanish]{babel}
\usepackage[utf8]{inputenc}
\usepackage[T1]{fontenc}
\usepackage{hyperref} % links en índice
\usepackage{tabularx} % tablas copadas
\usepackage{charter}   % tipografia
\usepackage{graphicx}
\usepackage{amsmath, amsthm, amssymb}
\usepackage{listings}
%\usepackage{makeidx}
\usepackage{paralist} %itemize inline
\usepackage[table]{xcolor}

\usepackage{float}
\usepackage{amsfonts}
\usepackage{sectsty}
\usepackage{charter}
\usepackage{wrapfig}

%\lstset{language=C}

% \setcounter{secnumdepth}{2}
\usepackage{underscore}
\usepackage{caratula/caratula}
\usepackage{url}


%para insertar algoritmos
\usepackage{xspace}
\usepackage{xargs}
\usepackage{algorithm}% http://ctan.org/pkg/algorithms
\usepackage{algpseudocode}% http://ctan.org/pkg/algorithmicx
\usepackage{verbatim}
\usepackage{listings}



% ********************************************************* %
% ~~~~~~~~              Code snippets             ~~~~~~~~~ %
% ********************************************************* %

\usepackage{color} % para snipets de codigo coloreados
\usepackage{fancybox}  % para el sbox de los snipets de codigo

\definecolor{litegrey}{gray}{0.94}

\newenvironment{codesnippet}{%
	\begin{Sbox}\begin{minipage}{\textwidth}\sffamily\small}%
	{\end{minipage}\end{Sbox}%
		\begin{center}%
		\vspace{-0.4cm}\colorbox{litegrey}{\TheSbox}\end{center}\vspace{0.3cm}}



% ********************************************************* %
% ~~~~~~~~         Formato de las páginas         ~~~~~~~~~ %
% ********************************************************* %

\usepackage{fancyhdr}
\pagestyle{fancy}

%\renewcommand{\chaptermark}[1]{\markboth{#1}{}}
\renewcommand{\sectionmark}[1]{\markright{\thesection\ - #1}}

\fancyhf{}

\fancyhead[LO]{Sección \rightmark} % \thesection\ 
\fancyfoot[LO]{\small{Joel Esteban Cámera, Alejandro Lavía, Martin Jonas}}
\fancyfoot[RO]{\thepage}
\renewcommand{\headrulewidth}{0.5pt}
\renewcommand{\footrulewidth}{0.5pt}
\setlength{\hoffset}{-0.8in}
\setlength{\textwidth}{16cm}
%\setlength{\hoffset}{-1.1cm}
%\setlength{\textwidth}{16cm}
\setlength{\headsep}{0.5cm}
\setlength{\textheight}{25cm}
\setlength{\voffset}{-0.7in}
\setlength{\headwidth}{\textwidth}
\setlength{\headheight}{13.1pt}

\renewcommand{\baselinestretch}{1.1}  % line spacing

% ******************************************************** %


\begin{document}


\thispagestyle{empty}
\materia{Organización del Computador II}
\submateria{Primer Cuatrimestre de 2016}
\titulo{Trabajo Práctico III - System Programming}
\subtitulo{Grupo: Más peligroso que mono con navaja}
\integrante{Joel Esteban Cámera}{257/14}{joel.e.camera@gmail.com}
\integrante{Alejandro Lavía}{43/11}{lavia.alejandro@gmail.com}
\integrante{Martin Jonas}{180/05}{martinjonas@gmail.com}

\maketitle
\newpage

%indice
\tableofcontents

\newpage

%\normalsize
\newpage

\section{Introducción}
Este trabajo presentado consiste en la construcción de un sistema mínimo en el cual aplicamos de forma gradual los conceptos de \textit{System Programming} vistos en las clases teóricas y prácticas.
Los ejercicios propuestos en este trabajo permiten que el sistema generado permita correr tareas a nivel usuario y que pueda ser capaz de capturar cualquier problema que puedan generar las tareas tomando las acciones necesarias para quitar estas del sistema, ademas de que puedan ser cargadas en el sistema dinamicamente por medio del uso del teclado.

El sistema resultante es un juego dentro del cual hay dos jugadores y estos tendran la posibilidad de selecciona una posición de la patnalla y lanzar una tarea infectada y estas pueden lanzar vectores que acceden a otras posiciones de memoria y así alterar el código de otras tareas.

Para el desarrollo del sistema contamos con el programa \textit{Bochs} que permite simular una computadora compatible que ejecuta nuestro sistema y, con respecto a los lenguajes de programación, hemos utilizado C y ASM según la rutina que estuviéramos desarrollando.
\newpage

\section{Desarrollo}

\subsection{Segmentación y Pasaje a Modo Protegido}

\subsubsection{Inicialización de la Tabla de Descriptores Globales (GDT)}
Lo primero que hemos hecho fue completar la GDT con 4 segmentos, dos para nivel 0, uno de código y uno de datos, y luego dos para nivel 3 tambien de codigo y datos. Estos segmentos fueron puestos en las posiciones 4 a 7 de la tabla, dejando libres las posiciones 1 a 3 de la tabla y la posición 0 con una entrada completamente en cero.

Los cuatro segmentos seteados direccionan a los primeros 878MB de memoria desde la posición 0. En este espacio entran 224.768 (0x36E00) bloques de 4KB. Como el límite del segmento es igual al tamaño del mismo menos 1, y que se accede desde la posicion 0, éste es 0x36DFF. Por lo tanto, en estos cuatro segmentos en la parte baja del límite ([15:0]) el valor es 0x6DFF y en la parte alta del mismo ([19:16]) 0x03. El bit de granularidad fue seteado en 0x01 (4K). El bit D/B fue seteado en 0x01 (Segmentos de 32-bit). El bit L fue seteado en 0x0 (No es segmento de 64-bit). El bit AVL fue seteado en 0x0 (ya que no tiene ningun uso especifico). El bit P fue seteado en 0x01 (los segmentos estan presentes). Los dos bits de DPL fueron seteados en 0x0 ó 0x3 según corresponda si el privilegio es sistema o usuario. El bit S fue seteado en 0x1 (desactivado). En los dos segmentos de código, los cuatro bits de Type fueron seteados en 0x8 (Execute-Only) y en los dos segmentos de datos fueron seteados en 0x2 (Read/Write).

Además de estos cuatro segmentos, se agrega uno más de datos en la posición 8 de la tabla que describe el área de la pantalla en memoria que puede ser utilizado solo por el kernel, esto quiere decir que el DPL del mismo esta seteado en 0x0. El cambio con los anteriores segmentos asignados en la GDT es la base y el límite. La base del mismo es 0xB8000, por lo tanto en la parte baja de la base (base[0:15]) se agrega el valor 0x8000 y en la parte alta de la misma (base[23:16]) 0xB. El límite del mismo es 0x1, por lo tanto en la parte baja del mismo (limit[0:15]) agregamos el valor 0x1 y en la parte alta (limit[16:19]) el valor 0x0.

\subsubsection{Pasaje a Modo Protegido}
Una vez que tenemos completada la GDT, deshabilitamos las interrupciones (\textit{CLI}) ya que no estan cargadas las rutinas de atencion de interrupciones. Luego habilitamos A20 (\textit{call habilitar_A20}) para que se habilite el acceso a direcciones superiores a los $2^{20}$ bits.

Una vez habilitada A20, cargamos la GTD en el registro GDTR con la dirección y el largo del mismo (\textit{lgdt [GDT_DESC]}) donde \textit{GDT_DESC} es un struct que posee estos datos.

Teniendo habilitada la A20 y cargado la GTD pasamos a modo protegido seteando en 1 el bit \textbf{PE} del registro de control \textbf{CR0}.

\begin{algorithm}
		\begin{algorithmic}
			\State mov eax, cr0
			\State or eax, 1
			\State mov cr0, eax
		\end{algorithmic}
\end{algorithm}

La instrucción inmediatamente siguiente es un \textit{far jump} a la siguiente instrucción, lo que permite cargar el registro \textbf{CS} con el selector del segmento del código del kernel (el 0x20 en nuestra GDT). \textit{jmp 0x20:en_modo_protegido}.

Luego establecemos los selectores de segmento de datos de máximo privilegio (el 0) y seteamos la base de la pila del kernel en la dirección 0x27000 ya que utilizaremos la pagina de 4k en 0x26000 como pila.

\begin{algorithm}[H]
		\begin{algorithmic}[H]
			\State ; Establecer selectores de segmentos
			\State xor eax, eax
			\State mov ax, 0x28
			\State ; Configuramos el selector de segmento de datos
			\State mov ds, ax
			\State ; Configuramos el selector de segmento de pila (stack)
			\State mov ss, ax

			\State ; Establecer la base de la pila
			\State mov ebp, 0x27000
			\State mov esp, ebp
		\end{algorithmic}
\end{algorithm}

Una vez realizado esto, inicializamos la pantalla de nuestro sistema con el siguiente código.

\begin{algorithm}[]
		\begin{algorithmic}[H]
			\State pantalla_limpia times 4000 db 176, 7
			\State pantalla_limpia_len equ   \$ - pantalla_limpia
			\State ; Inicializar pantalla
			\State mov esi, pantalla_limpia
			\State mov edi, 0x0
			\State mov ax, 0x40
			\State mov es, ax
			\State mov ecx, 4000
			\State rep movsw
		\end{algorithmic}
\end{algorithm}

\subsection{Interrupciones y Excepciones}

\subsubsection{Inicialización de la IDT (Interrupt Descriptor Table)}

En el arreglo de \textbf{itd_entry} de 255 (IDT), cargamos las interrupciones que necesitan ser manejadas, para esto configuramos las entradas como \textit{interrupt gate}, para manejar las interrupciones y excepciones del procesador (valores 0 a 14 y 16 a 19), tamben registramos la interrupcion 32 (reloj), 33 (teclado) y 102 utilizada arbitrariamente como syscall de nuestro sistema.
Para inicializar esta tabla, realizamos un call a la función \textbf{itd_inicializar} que esta definida en \textbf{idt.c}. Esta función llama al macro \textbf{IDT_ENTRY} para configuar las entradas correspondientes. 
Todas las entradas fueron seteadas de esta forma, salvo la 102 que haremos la mención de la diferencia en donde corresponda:

\begin{itemize}
	\item Segment Selector: 0x20. El selector de segmento de código del Kernel.
	\item Offset: \&_isr + (número de interrupción). En el espacio de código de Kernel que corra la rutina de atención.
	\item P: 0x01. Presente.
	\item DPL: 0x03 para la interrupción 102 (para poder ser llamada por las tareas), y 0x00 para el resto de las interrupciones.
	\item Atributos: 0b0111000000000
\end{itemize}

%Todas las interrupciones se configuran con privilegio maximo excepto la 102 que se configura con privilegio 3 para asi poder ser llamada por las tareas en el momento de ejecutar un syscall.


\subsubsection{Carga de la IDT}

La carga de esta tabla es similar a la carga de la GDT, poseemos el struct \textit{IDT_DESC} que posee el límite y la dirección de base de la tabla y, con la función \textbf{lidt} la cargamos (\textit{lidt [IDT_DESC]}).


\subsubsection{Rutina de Atención de Interrupciones}

Las rutinas de atencion estan definidas en \textbf{isr.asm} y se comportan de manera generica para las interrupciones y excepciones del procesador, luego hay funciones especialmente definidas para atender el reloj, el teclado y las syscall.\\
La rutina de atencion del reloj, se encarga de llamar al scheduler para obtener cual es la proxima tarea a ejecutar y saltar a dicha tarea.\\
La rutina de atencion del teclado, se encarga de llamar a una funcion en C \textbf{keyPress} que esta definida en \textbf{game.c} y se encargara dependiendo de la tecla precionada que operacion realizar.\\
La rutina de atencion del syscall realiza un push de los 3 posibles parametros enviados a una de estas rutinas y llama \textbf{my_syscall} tambien definida en \textbf{game} y que se encargar interpretar los parametros, validarlos y realizar las operaciones necesarias.

\subsection{Paginación}

\subsubsection{Inicialización del Directorio y Tabla de Páginas (Page Directory y Page Table)}

Antes de activar paginación primero generamos el directorio de páginas (Page Directory) con la función \textbf{mmu_inicializar_dir_kernel} que se encuentra en el archivo \textbf{mmu.c}. Esta función, a su vez, llama a  \textbf{inicializar_identity_mapping} pasandole como parametros \textit{KERNEL_PAGE_DIRECTORY} y \textit{KERNEL_PAGE_TABLE} que son las posiciones 0x27000 y 0x28000 respectivamente.
La función \textbf{inicializar_identity_mapping} se va encargar de inicializar como identity mapping los primeros 4mb de memoria.

\subsubsection{Activación de la Paginación}

Una vez que fueron generadas las tablas, activamos la paginación en el código del kernel de la siguiente forma:

\begin{algorithm}[H]
		\begin{algorithmic}[H!]
			\State ; Inicializar el directorio de paginas
    		\State call mmu_inicializar_dir_kernel

			\State; Cargar directorio de paginas
    		\State mov eax, 0x27000
    		\State mov cr3, eax

    		\State ; Habilitar paginacion
    		\State mov eax, cr0
    		\State or eax, 0x80000000
    		\State mov cr0, eax

		\end{algorithmic}
\end{algorithm}


\subsubsection{Administrador de Memoria}

Para administrar la memoria en el area libre, creamos la función llamada \textbf{mmu_inicializar} que simplemente inicializa la variable global \textit{proxima_pagina_libre} que se encuentra en el archivo \textbf{mmu.c}.
\textit{proxima_pagina_libre} es un contador de páginas que comienza en la dirección 0x100000. 
Cuando el sistema necesita memoria, toma el valor de este contador, que corresponde a la siguiente página libre y luego lo incrementa en 0x1000 que es el tamaño de cada página.


\subsubsection{Mapeado de Páginas}

Para el mapeado de páginas contamos con la función \textbf{mmu_mapear_pagina} que toma como parametros la direccion virtual, el cr3 y la direccions fisica y el privilegio (DPL). En la dirección virtual se shiftea hacia la derecha para dejar los valores de los bits [31:22] para obtener el offset del Page Directory. Una vez obtenido el offset se lo suma a la dirección que se encuentra en el registro cr3 para obtener de la page directory la entrada de la page table y ahí buscar la dirección fisica. Si ésta no está presente, se reserva una pagina de memoria utilizando \textbf{mmu_proxima_pagina_fisica_libre} y se genera una nueva entrada de página en el lugar offset de la Page Directory con la función \textbf{inicializar_page_entry}. 
Luego, de la dirección virtual extraemos los bits [21:12] (con otro shift y limpiando la parte alta) y utilizando nuevamente inicializar_page_entry generamos en la tabla de páginas, con el offset obtenido una entrada presente, de escritura y con el DPL y direccion fisica pasados como argumento. Por último, se llama a la función \textbf{tlbflush} para que invalide el \textit{cache} de traducción de direcciones.

\subsubsection{Desmapeado de Páginas}

Para el desmapeado de las páginas contamos con la función \textbf{mmu_unmapear_pagina}, esta toma como parametros la dirección virtual y el cr3. De la dirección virtual obtenemos el offset del directorio de páginas y de la tabla de páginas. Luego, en el page directory apuntado por cr3 buscamos la page table, y en la page table ponemos todos los bits en cero para la entrada obtenida de la direccion virtual. Una vez hecho esto, se llama a la función \textbf{tlbflush} para que invalide el \textit{cache} de traducción de direcciones.


\subsubsection{Inicialización de Páginas Para Tareas}

Creamos la función \textbf{mmu_inicializar_dir_tarea} que lo que hace es reservar para cada tarea dos paginas libres, con la función \textbf{mmu_proxima_pagina_fisica_libre}, la primera página se utiliza para la creacion del directorio de pagina que utilizara la tarea nueva, la seguna pagina se utiliza como page table inicializado con identity mapping los primeros 4mb de memoria con privilegio 0, para que cuando haya un cambio de contexto, por una interrupcion durante la ejecucion de esta tarea, el kernel tenga mapeado el mismo area de memoria.
Luego se realiza un identity mapping temporal,  sobre el directorio de paginas de la tarea que se esta ejecutando al llamar a la interrupcion, para asi poder copiar la tarea que se desea lanzar, desde el origen a la posicion del mapa requerida.
Luego se mapea para la nueva tarea, la direccion 0x08000000 con privilegio de usuario a la posicion fisica del mapa, ademas se realiza el identity mapping con privilegio kernel a esa direccion fisica, para que el kernel pueda por ejemplo escribir el resultado del syscall 'donde'.

\subsection{Interrupciones Externas}

\subsubsection{PIC}

Teniendo ya hecha la tabla \textbf{IDT} y generadas las rutinas de atención correspondientes a las excepciones internas del procesador, aquí sólo cargamos las tres interrupciones externas que puede recibir el sistema: el reloj, el teclado y una rutina que provee los servicios de servicios del sistema. Las interrupciones del reloj y teclados son pedidas por el \textbf{PIC} y la última es pedida por el usuario.

El \textbf{PIC} puede atender 15 interrupciones (IRQ0 - IRQ15, la IRQ2 no cuenta ya que es donde se conecta otro \textbf{PIC} en cascada). Por defecto, estas IRQs están mapeadas a las interrupciones 0x8 a 0xF (\textbf{PIC1}) y de 0x70 a la 0x77 (\textbf{PIC2}) pero las interrupciones de la 0 a la 31 están reservadas para el procesador y, en particular, de la 8 a la 15 ya están ocupadas por las excepciones del mismo. Cuando se produzcan interrupciones, se llama al handler de la excepción. Por esto, hay que ``remapear" { } las interrupciones del \textbf{PIC}. Para esto, utilizamos las funciones provistas por la catedra \textbf{resetear_pic} y \textbf{habilitar_pic} que las llamamos dentro del código del kernel.

\begin{algorithm}[]
		\begin{algorithmic}[H]
			\State call resetear_pic
    		\State call habilitar_pic
		\end{algorithmic}
\end{algorithm}

Luego de remapear el \textbf{PIC} y habilitarlo, tenemos que la interrupción de reloj está mapeada a la interrupción 32 y el teclado a la 33.

\subsubsection{Interrupción de Reloj}

El código utilizado para atender esta rutina es el mismo que el dado en la clase práctica de scheduler pero con algunas líneas más que se agregaron para poder implementar el modo debug.

Lo primero que realiza esta interrupción es la de guardar en la pila todos los registros con la función \textbf{pushad}. Luego, revisa si esta en modo debug, si lo está salta al final de la rutina (\textbf{.nojump}), sin modificar nada del juego, ejecutando la tarea idle.
Si no lo está sigue el mismo código dado en clase: llama a proximo_reloj y luego llama a sched_proximo_indice que calcula que tarea se debe ejecutar. Si el indice es el mismo que el actual salta al final de la interrupción terminandola. Si no lo es, mueve el índice al selector \textit{sched_tarea_selector}, resetea el \textbf{PIC} y ejecuta el far jump al offset \textit{sched_tarea_offset} para hacer el cambio de tarea.


\subsubsection{Interrupción de Teclado}

En esta interrupción se lee el teclado a través del puerto 0x60 y se obtiene el scan code. Si se presiono la tecla 'y' se entra en modo debug y se sale de la interrupción, sino revisa si esta en modo debug y si se apreto nuevamente la tecla 'y' para salir de él o, si no cumple todas estas condiciones pushea el valor de la tecla apretada y llama a la función \textbf{keyPress} (que se encuentra en \textbf{game.c}) en donde si es una de las teclas permitidas del juego mueve al jugador correspondiente o lanza la tarea.

\subsubsection{Servicios del Sistema}

Dentro de esta interrupción, se pushean los registros eax, ebx y ecx (si bien no en todas las funciones se requieren los 3 argumentos) y se llama a la función \textbf{my_syscall} que se encuentra en el archivo \textbf{game.c}. Luego, se realiza un far jmp a la tarea idle y se termina la interrupción.

Dentro de \textbf{my_syscall} se revisa el parametro eax y luego, dependiendo del valor de ese registro, pasa los valores ebx o, ebx y ecx a las funciones \textbf{game_donde}, \textbf{game_soy} o \textbf{game_mapear} que se encargan de solicitar la posición x e y del mapa donde se encuentra mapeada la tarea, informa al sistema si la correspondiente tarea esta infectada o no y solicita mapear su pagina virtual a una pagina física en el mapa dada por las coordenadas x e y respectivamente.

\subsection{Tareas}

\subsubsection{Task-State Segment}

Nuestro sistema posée 27 tareas entre las cuales se encuentran la tarea inicial sin valores especificados (ya que solo cumple la funcion de guardar el estado anterior a saltar a la tarea idle), la tarea idle, 15 tareas health, 5 tareas del jugador A y 5 tareas del jugador B. Para esto creamos un arreglo de 27 entradas (llamado \textbf{tss_entries}) y el mismo se encuentra en el archivo \textbf{tss.c}) en las que están cada una de las tss de las tareas.

Hemos distribuído el arreglo de la siguiente forma para tener un mejor control sobre las TSS:

\begin{itemize}
	\item Posición 0: TSS tarea inicial.
	\item Posición 1: TSS tarea idle.
	\item Posiciones 2 a 16: TSS tareas health.
	\item Posiciones 17 a 21: TSS tareas jugador A.
	\item Posiciones 22 a 26: TSS tareas jugador B.
\end{itemize}

Para la creación de cada TSS hicimos las funciones \textbf{tss_inicializar_idle}, \textbf{tss_inicializar_tarea} y \textbf{inicializar_tss} donde las primeras dos llaman a la última que es la que setea todos los valores de la TSS de la siguiente forma:

\begin{itemize}
	\item Registros en 0.
	\item SS0: Segmento de datos de nivel kernel.
	\item ESP0: La dirección de una nueva pagina reservada para funcionar como pila de kernel más el page size.
	\item CR3: Dirección del page directory de la tarea ó 0x27000 para idle.
	\item EIP: Posicion de la tarea idle dentro del espacio de kernel ó la direccion 0x08000000 que es la direccion virtual de inicio de las tareas.
	\item FLAGS: 0x202 (Interrupciones activadas).
	\item ESP/EBP: Para la tarea idle la misma dirección que en ESP0. Para la tarea EIP más el page size (0x1000).
	\item Segmentos: Segmento de datos kernel para idle y segmento de datos de nivel usuario para las tareas con un or 3 para setear el RPL del selector de segmento.
	\item CS: Segmento de código de nivel kernel par idle y de usuario con un or 3 para setear el RPL para las tareas.
	\item Resto de los valores en cero.
\end{itemize}

\subsubsection{Entradas En La GDT}

En la GDT definimos 27 entradas de descriptores de TSS de las cuales la primera posición pertenece a la tarea inicial, la segunda posición pertenece a la tarea idle y las siguientes son las tareas que el sistema corre concurrentemente.
Estas fueron seteadas de la siguiente forma:

\begin{itemize}
	\item Base: Dirección de la TSS. (Calculada usando el vector de TSSs \&tss_entries[indice_tss])
	\item Limite: 103 (el tamaño menos 1 de la TSS)
	\item Type: 1001 (32-bit TSS)
	\item System: 0
	\item DPL: 00 (Kernel)
	\item AVL: 0
	\item L: 0
	\item P: 1
	\item DB: 1 (32-bit Segment)
	\item G: 0 (Byte)
\end{itemize}

Para inicializar los valores en la GDT creamos la función \textbf{inicializar_gdt_tss} que se encuentra en el archivo \textbf{gdt.c} y llamamos desde el código del Kernel. Esta función genera las 27 entradas en la GDT de la forma descripta.

\subsubsection{Salto a la primera tarea (idle)}

Una vez que se tienen creadas las TSSs y los descriptores de TSS en la GDT, se carga el descriptor de la tarea inicial, y luego se realiza un \textbf{jmp far} a la tarea idle.

\begin{algorithm}
		\begin{algorithmic}[H]
			\State ; Cargar tarea inicial
			\State mov eax, 0x48
			\State ltr ax
			\State
			\State ; Saltar a la primera tarea: Idle
			\State jmp 0x50:0
		\end{algorithmic}
\end{algorithm}

\subsection{Scheduler Y Juego}


\subsubsection{Estructura Del Scheduler}

Para el scheduler generamos la estructura \textbf{TAREA_INFO}, que se encuentra en el archivo \textbf{sched.h}. En esta estructura se puede encontrar:

\begin{itemize}
\item char \textit{viva}: Define si la tarea esta viva o no.
\item short \textit{pagina_x} y \textit{pagina_y} : Son las coordenadas en el mapa (y permiten deducir la posicion en la memoria) de la pagina mapeada por la tarea (si lo hubiese hecho).
\item char \textit{estado}: nos dice si la tarea esta health o si esta infectada por a o b.
\item int \textit{tssIdx}: Indica la tarea real apuntada por esta entrada del vector. (es el indice en el vector \textbf{tss_entries})
\item char \textit{tick}: se va actualizando con un \% 4 y sirve para el reloj de la tarea.
\end{itemize}

De esta estructura creamos 3 arreglos para los tres distintos tipos de tareas que tenemos: \textbf{tareas_health}, \textbf{tareas_a} y \textbf{tareas_b}. Estos se encuentran en \textbf{sched.c} y nos sirven para saber que tareas estan vivas, cuales estan infectadas y el orden en que van a ser ejecutadas.

Tambíen tenemos cinco enteros: \textbf{actual_health}, \textbf{actual_a}, \textbf{actual_b}, \textbf{tssActual} y \textbf{actual_tipo}. Los primeros tres sirven para saber cual es la tarea de cada arreglo que está corriendo. Los últimos dos sirven para saber el offset en la GDT del descriptor de TSS de la tarea que esta corriendo y el tipo de la tarea que esta corriendo (a, b o health).


\subsubsection{Funcionamiento Del Scheduler}

Desde el codigo del Kernel llamamos a la función \textbf{inicializar_scheduler} en donde se inicializan todos los valores de la estructura del mismo. Los enteros se incializan todos en cero y los arreglos de tareas sin tareas vivas, luego se llamara a \textbf{inicializar_juego} que creara las 15 tareas health.

Cuando un jugador decide lanzar una tarea, utilizando la función \textbf{lanzarTarea} que se encuentra en el archivo \textbf{sched}, siempre y cuando de acuerdo a la logica del juego pueda hacerlo, se busca la proxima posición libre del arreglo del jugador que la lanzó, o sea una posición que tenga una tarea que no esté viva, la setea como viva y devuelve el indice de la TSS en el arreglo \textbf{tss_entries}.

El \textit{Quantum} de cada tarea es el un clock del reloj. Por lo tanto, cada vez que se atienda una interrupción de reloj dentro de esta rutina se llama a la función \textbf{sched_proximo_indice} (que se encuentra en el archivo \textbf{sched.c}). Esta función lo que hace es actualizar el valor \textbf{actual_tipo} (con un modulo 3 ya que hay solo tres tipos de tareas) e ir a buscar al arreglo del tipo (0 = health, 1 = jugador a y 2 = jugador 3) la siguiente tarea a correr. Si la encuentra, devuelve el indice del descriptor de TSS de la tarea en la GDT sino devuelve el indice del descriptor de TSS de la tarea idle en la GDT, y con estos indices realiza el \textit{jmp far}.

Si mientras se ejecuta una tarea se produce una excepcion, al llamar a la rutina de atencion de interrupcion de esa excepcion se mata la tarea que estuvo corriendo y salta a la tarea idle para que corra hasta que se termine el \textit{Quantum} (cuando haya una interrupción de reloj) de la tarea que origino la excepcion . Para matarla llama a la función \textbf{game_matar_tarea}, que se encuentra en \textbf{game.c}. Dentro de esta función se actualizan las estructuras de manejo de tareas, luego llama a la función \textbf{matarTarea}, que se encuentra en \textbf{sched.c}, en donde pone a la tarea como muerta dentro de su arreglo de tareas dependiendo de que tipo sea (health, a o b).
Luego, se desmapea la pagina de la tarea.


\subsubsection{Juego}

% Para el manejo de las tareas utilizamos una matriz del tamaño del mapa para guardar información sobre las tareas de cada posición, a su vez tres vectores para manejar la información de las tareas y dos estructuras para manejar la información de los jugadores.

% Queda pendiente una explicación detallada sobre el funcionamiento de estas estructuras.

Para manejar la informacion del juego utilizamos una matriz de 80x44 (tamaño del mapa) llamada \textbf{mapa} (que se encuentra en \textbf{screen.c}). Las entradas de esta matriz tienen un struct que posee el indice de la TSS en el vector \textbf{tss_entries}, un char \textit{dirty}, un char \textit{infectado_por}, un char \textit{tarea_de} y otro char \textit{pagina_de}.
Esta matriz \textbf{mapa} nos sirve para saber que tarea hay en cada posición en el mapa, si fueron infectadas, de quien era originalmente y si esa posicion fue mapeada por alguna tarea. El char \textit{dirty} sirve para saber si hubo una modificación en esa coordenada del mapa y poder actualizar la pantalla solo en la posicion donde hubo modificaciones sin necesidad de tener que actualizar todo en cada ciclo.

Para los jugadores creamos un struct que posee las coordenadas del mapa en donde se encuentra cada uno y la cantidad de vidas, puntos y tareas activas que tiene el jugador. Utilizamos esta estructura en \textbf{screen.c} para los dos jugadores del juego.


\subsection{Modo Debug}

El modo debug lo implementamos con un flag llamado \textbf{'modo_debug'} de un byte en la rutina de atención de interrupciones, este flag puede estar en tres estados diferentes.

\begin{itemize}
\item \textbf{0x00}: Corresponde al modo debug desactivado.
\item \textbf{0x01}: Corresponde al modo debug activo.
\item \textbf{0x02}: Corresponde al modo debug cuando esta mostrando una excepción por pantalla.
\end{itemize}

En la rutina de atención de interrupción del teclado \textbf{isr33} agregamos el código encargado de alternar entre los estados de debug al capturar la tecla 'y', lo que hace es alternar entre modo debug activado y desactivado, o si estaba mostrando una excepción, dejar de mostrarla y ponerse en activo nuevamente a la espera de una nueva excepción.
\vspace{0.25cm}

En las rutinas encargadas de manejar las excepciones se agrega una comprobación para fijarse el modo de debug actual, en caso de estar activo como se acaba de generar una excepción se llama a la rutina \textbf{'mostrarExcepcion'} encargada de mostrar por pantalla toda la información de debug, para eso esta rutina pone en la pila toda la información adicional necesaria y llama a una función de \textbf{screen.c} para pintar el recuadro donde se mostrara, luego se saltara a la tarea idle.
\vspace{0.25cm}

Finalmente para mantener el sistema a la espera mientras se muestra la excepción, en la rutina de atención de reloj \textbf{isr32} comprobamos si estamos efectivamente mostrando una excepción por pantalla, en ese caso ignoramos al scheduler y nos mantenemos en la tarea idle, así estaremos hasta que una interrupción de teclado con la tecla 'y' nos saque de este modo y se resuma el funcionamiento del scheduler.


% \section{Resultados} 
% \input{resultados}
% \newpage

%\begin{figure}
%  \begin{center}
%	\includegraphics[scale=0.66]{imagenes/logouba.jpg}
%	\caption{Descripcion de la figura}
%	\label{nombreparareferenciar}
%  \end{center}
%\end{figure}


%\paragraph{\textbf{Titulo del parrafo} } Bla bla bla bla.
%Esto se muestra en la figura~\ref{nombreparareferenciar}.



%\begin{codesnippet}
%\begin{verbatim}

%struct Pepe {

%    ...

%};

%\end{verbatim}
%\end{codesnippet}

\end{document}

