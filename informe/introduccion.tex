Este trabajo presentado consiste en la construcción de un sistema mínimo en el cual aplicamos de forma gradual los conceptos de \textit{System Programming} vistos en las clases teóricas y prácticas.
Los ejercicios propuestos en este trabajo permiten que el sistema generado permita correr tareas a nivel usuario y que pueda ser capaz de capturar cualquier problema que puedan generar las tareas tomando las acciones necesarias para quitar estas del sistema, ademas de que puedan ser cargadas en el sistema dinamicamente por medio del uso del teclado.

El sistema resultante es un juego dentro del cual hay dos jugadores y estos tendran la posibilidad de selecciona una posición de la patnalla y lanzar una tarea infectada y estas pueden lanzar vectores que acceden a otras posiciones de memoria y así alterar el código de otras tareas.

Para el desarrollo del sistema contamos con el programa \textit{Bochs} que permite simular una computadora compatible que ejecuta nuestro sistema y, con respecto a los lenguajes de programación, hemos utilizado C y ASM según la rutina que estuviéramos desarrollando.